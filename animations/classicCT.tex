\documentclass{standalone}
% Draw the setup where the source and detector move, e.g. classic CT
% With help from https://tex.stackexchange.com/q/515519/828
\usepackage{animate}
\usepackage{tikz}
%	\usetikzlibrary{external}
%	\tikzexternalize
%	\tikzsetnextfilename{classicCT}
\usepackage{fontawesome5}	
\usepackage{ifthen}
\ifthenelse{\isundefined{\everyframe}}{%
	% If we're compiling this file via \input, then these variables are already defined.
	% In the other case, we need to define them
	\newcommand{\everyframe}{5}
	\definecolor{ubRed}{HTML}{E6002E}
	% split complementary images from https://www.sessions.edu/color-calculator/
	\definecolor{ubRedComplementary1}{HTML}{00a1e6}
	\definecolor{ubRedComplementary2}{HTML}{00e645}
	\definecolor{ubGrey}{RGB}{217,217,217}
	}{}
\begin{document}
\begin{animateinline}[every=\everyframe]{25}
	\multiframe{180}{n=1+2}{%
%	\tikzifexternalizing{Work-around to make animate happy	}{}%https://tex.stackexchange.com/a/39026/828
		\begin{tikzpicture}
			\pgfdeclarelayer{background}
			\pgfsetlayers{background,main}
			%Help lines
			\draw[<->] (-2.25,0) -- (2.25,0);
			\draw[<->] (0,-2.25) -- (0,2.25);
			\draw[help lines,step=1cm,ultra thin] (-2.45,-2.45) grid (2.45,2.45);
			% Stuff that stays put
			\node[ubRedComplementary1] at (0,0) (sample) {\Huge\faUser};
			% Stuff that moves
			\mode<beamer>{%
				\begin{scope}[rotate around={\n:(sample)}]
				}
				% Rotation arc
				\draw[->, thick,line cap=rect] (1.5,0) arc [start angle=0, end angle=180, radius=1.5];
				\draw[->, thick,line cap=rect] (-1.5,0) arc [start angle=-180, end angle=0, radius=1.5];
				% Source
				\fill[ubRed] (-0.25,1.5) rectangle node (source) [black] {Röntgenquelle} +(0.5,0.5);
				% Detector and detector edges
				\fill[ubRedComplementary2,fill] (-0.5,-1.75) rectangle node (detector) [black] {Detektor} +(1,0.25);
				\coordinate (dl) at (-0.45,-1.75);
				\coordinate (dr) at (0.45,-1.75);
				% X-ray cone
				\begin{pgfonlayer}{background}
					\fill[gray,semitransparent] (source.center) -- (dl) -- (dr) -- cycle;
				\end{pgfonlayer}
			\mode<beamer>{%
				\end{scope}
			}
		\end{tikzpicture}
	}
\end{animateinline}
\end{document}
